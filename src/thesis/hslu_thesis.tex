%%%%%%%%%%%%%%%%%%%%%%%%%%%%%%%%%%%%%%%%%
%
% (c) Lucerne University of Applied Sciences and Arts
%
% HSLU-I: Official Thesis Template
%
% This template complies with the official thesis guidelines released
% on complesis by the department HSLU-I (computer science).
% It supports the languages english and german. The language and thesis style
% can be set by the parameters passed to the documentclass on line 33.
%
% NOTE: The class and style sheets (hsluthesis.cls, hsluthesisterms.sty) should
%       NOT be changed. These are configuration files and define the document style.
%
% Original guidelines can be found on complesis:
% https://complesis.hslu.ch/
% 
% Original author:
%   Ramón Christen HSLU-I
%
% Versions:
%   v1.0    21-03-2025: Initial Version
%   v2.0    22-09-2025: Reviewed template structure (.cls added); Extension for continuous education
%   v2.1    29-09-2025: Extended for WIPRO
%   v2.2    10-10-2025: Minor corrections for WIPRO
%
%%%%%%%%%%%%%%%%%%%%%%%%%%%%%%%%%%%%%%%%%


%----------------------------------------------------------------------------------------
%	PACKAGES AND DOCUMENT CONFIGURATIONS
%----------------------------------------------------------------------------------------
% class options (set in \documentclass[...]; order of options is irrelevant):
%   degree:       [wipro, bachelor, master, cas, sas]
%   language:     [english, german]
%   groupwork:    [groupwork] (extends declaration for two students)
%   ip:           [noip] (only for cas/sas if no intellectual property consent is required)
% 
% Examples for configuring the documentclass:
% \documentclass[german,master]{hsluthesis}
% \documentclass[english,bachelor]{hsluthesis} 
% \documentclass[german,wipro,groupwork]{hsluthesis} 
% \documentclass[english,cas,noip]{hsluthesis} 
\documentclass[english,bachelor]{hsluthesis} 


\usepackage{comment}                            % having comment sections \begin{comment} \end{comment}
\usepackage{amsmath}							% math package
\usepackage{amsfonts}							% font package for math symbols
\usepackage{amssymb}							% symbols package - definition of math symbols
\usepackage{listings}							% package for code representation
\usepackage{graphicx}							% for inclusion of image
\usepackage{subfig}								% to arrange figures next to each other
\usepackage{float}								% text style surrounding images
\usepackage[acronym]{glossaries}         		% package for glossary
\usepackage{tikz}								% used to place logos on title page
% \usepackage{gensymb}							% for special characters such as °
\usepackage[a-1a]{pdfx}                         % Forces PDF/A-1a compliance for long-term archiving


\usepackage{multirow}
\usepackage{siunitx}
\usepackage{tabularx}
\usepackage{tikzscale}


\hypersetup{hidelinks}                          % hide red border in hyperlinks
\setcounter{tocdepth}{1}                        % hide subsections from TOC
\makenoidxglossaries
\input{acronyms}                                % include acronyms.txt file
\input{glossary}                                % include glossary.txt file
\graphicspath{{figs/}}						    % set path of graphics folder


%----------------------------------------------------------------------------------------
%	PDF/A DOCUMENT COMPLIANCE
%----------------------------------------------------------------------------------------
\pdfcatalog{
  /StructTreeRoot <<                            % Define the structure tree root for document tagging
    /Type /StructTreeRoot                       % Specify that this is a structure tree root
    /K []                                       % Placeholder for structure elements (empty for now)
  >>                
  /MarkInfo << /Marked true >>                  % Ensure the document is marked as tagged for accessibility
}


\begin{document}
%----------------------------------------------------------------------------------------
%	DOCUMENT INFORMATION
%----------------------------------------------------------------------------------------
% \thesisLanguage{english}                        % set thesis language [english, german]
\author{Author Name}                            % author name
\city{Lucerne (Switzerland)}                    % author's place of origin
\title{Thesis Title}                            % thesis title
\subtitle{\large subtitle}                      % thesis subtitle

\date{2025}                                     % the year when the thesis was written (used in titlepage)
\defensedate{October 27th, 2025}                % the date of the private defense
\defencelocation{Lucerne}                       % location of defence
\extexpert{Expert Name}                         % name of external expert
\indpartner{Company Name}                       % name of industry partner
\studyprogram{Study Program}                    % name of study program: Business Information Technology, International IT Management, ...

% jury, supervisor and dean are only relevant if acceptance sheet is enabled with the next line
% \addAcceptsheet
\jury{                                          % members of the jury
    \begin{itemize}
        \item Prof. Dr. Name Surname from Lucerne University of Applied Sciences and Arts, Switzerland (President of the Jury);
        \item Prof. Dr. Name Surname from Lucerne University of Applied Sciences and Arts, Switzerland (Thesis Supervisor);
        \item Prof. Dr. Name Surname from Lucerne University of Applied Sciences and Arts, Switzerland (External Expert).
    \end{itemize}
}

\supervisor{Prof. Dr. Name Surname}             % name of supervisor
\dean{Prof. Dr. Name Surname}                   % name of faculty dean

\acknowledgments{Thanks to my family, relatives and firends for all the support given to finish this thesis.}



%----------------------------------------------------------------------------------------
%	BEGIN DOCUMENT AND CREATE TITLEPAGE
%----------------------------------------------------------------------------------------
\maketitle


%----------------------------------------------------------------------------------------
%	PREAMBLE
%----------------------------------------------------------------------------------------
\begin{abstractstyle}{\hsummary}
    The content of your thesis in brief.
\end{abstractstyle}

\tableofcontents
\listoffigures
\listoftables
% print list of acronyms and glossary
\printnoidxglossaries



%----------------------------------------------------------------------------------------
%	MAIN CONTENT
%----------------------------------------------------------------------------------------
\mainmatter

% examplary content: write or compose the main document here
\chapter{Main Content}
This is a template of \gls{hslu} and then. This section usually comprises different chapters and subchapters.

\section{First Section}
Followed by a brief introduction, the section may comprise several subsections explaining various concepts and refering to external results \cite{christen_exogenous_2020}.

\subsection{First Subsection}
Content may also refer to special expressions such as that has to be explained in a separate section. Here we also have \gls{fuzzy logic} enough space and \gls{cww} to discuss the concept of \gls{nn} and so on.

\begin{comment}
    This is a sample comment.
\end{comment}

\chapter{Research Questions}
A relevant part of every scientific thesis build the formulation and declaration of research questions. These should be kept as clear and straightforward as possible, despite as preceise as required for scientific analysis. In this thesis, research questions can properly listed with the command:
\[\backslash resq\{number\}\{question\}\]

\resq{1}{How many research questions will be answered in this thesis?}
\resq{2}{Does there exist an easier way to declare research questions?}
\resq{3}{Is it feasible to answer all these questions?}

\bibliographystyle{ieeetr}
\footnotesize\bibliography{references}



%----------------------------------------------------------------------------------------
%	APPENDIX
%----------------------------------------------------------------------------------------
\appendix

\glsaddallunused                                % add all unused items to glossary

\end{document}
